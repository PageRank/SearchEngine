\documentclass[pdftex,12pt,a4paper]{article}

\usepackage[T1]{fontenc}
\usepackage[utf8]{inputenc}
\usepackage[lined,boxed,linesnumbered,commentsnumbered]{algorithm2e}
\usepackage{amsmath}


\begin{document}

\title{
	\Large \textbf{Algorithms for social networks} \\
	\textsf{Report of the project \emph{PageRank}}
}
\author{Andreea Beica \and Baptiste Lefebvre}
\date{21 januar 2014}

\maketitle


\section{Presentation}

Bla, bla, bla...


\section{Project architecture}

Bla, bla, bla...


\section{Algorithms}

Three methods have been selected to implement the PageRank algorithm:
\begin{itemize}
\item iterative method
\item algebraic method
\item power method
\end{itemize}
This section details each of this method. They are all three based on the same input, a set $P = p_1, p_2, ..., p_n$ of html pages, with some hyperlinks between pages (i.e. $p_i \rightarrow p_j$), which defines a webgraph $G = \left(V, E\right)$ such that:
$$V = \{i \in N / p_i \in P\}$$
and:
$$E = \{\left(i, j\right) \in V \times V / p_i \rightarrow p_j\}$$
To make notations convenient, the following concepts are useful:
\begin{itemize}
\item $M\left(v\right)$ denotes the set of vertices that link to $v$ in $G$
\item $L\left(b\right)$ denotes the number of outbound edges for node $v$ in $G$
\item $A$ denotes the adjacency matrix of $G$ (i.e. $A_G\left[i\right]\left[j\right] \in \{0, 1\}$ and $A_G\left[i\right]\left[j\right] = 1 \Leftrightarrow \left(i, j\right) \in E$)
\end{itemize}

\subsection{Iterative method}
\IncMargin{1em}
\begin{algorithm}[H]
\SetKwInOut{Input}{Input}
\SetKwInOut{Output}{Output}
\BlankLine
\Indm
\Input{A webgraph $G$ with $n$ nodes}
\Output{An array $pr$ of size $n$}
\Indp
\BlankLine
\emph{// Initialization}\;
\For{$i \leftarrow 1$ \KwTo $n$}{
$pr'\left[i\right]\leftarrow\infty$\;
$pr\left[i\right]\leftarrow\frac{1}{n}$\;
}
\emph{// Iterations}\;
\While{$|pr-pr'|\geq\epsilon$}{
$pr' \leftarrow pr$\;
\For{$i \leftarrow 1$ \KwTo $n$}{
$pr\left[i\right]\leftarrow\frac{1-d}{N}+d\sum_{j\in M\left(i\right)}\frac{pr'[j]}{L\left(j\right)}$\;
}
}
\emph{// Renormalization}\;
$pr\leftarrow\frac{pr}{|pr|}$\;
\BlankLine
\caption{Iterative methods}\label{algo_iterative_method}
\end{algorithm}
\DecMargin{1em}

\subsection{Algebraic method}
Additional notations have been used to describe the algebraic method, for a given integer $n$:
\begin{itemize}
\item $I$ denotes the identity matrix of size $n$
\item $O$ denotes the column vector of length $n$ containing only ones
\end{itemize}

\IncMargin{1em}
\begin{algorithm}[H]
\SetKwInOut{Input}{Input}
\SetKwInOut{Output}{Output}
\BlankLine
\Indm
\Input{A adjacency matrix $A$ of a webgraph $G$ with $n$ nodes}
\Output{An array $pr$ of size $n$}
\Indp
\BlankLine
\emph{// Initialisation}\;
\For{$j \leftarrow 1$ \KwTo $n$}{
\If{$L\left(j\right) \neq 0$}{
\For{$i \leftarrow 1$ \KwTo $n$}{
$A\left[i\right]\left[j\right]\leftarrow \frac{A\left[i\right]\left[j\right]}{L\left(j\right)}$
}
}
}
\emph{Computation}\;
$pr\leftarrow \left(dA-I\right)^{-1}\frac{1-d}{n}O$
\BlankLine
\caption{Algebraic method}\label{algebraic_method}
\end{algorithm}
\DecMargin{1em}


\subsection{Power method}
\IncMargin{1em}
\begin{algorithm}[H]
\SetKwData{Left}{left}\SetKwData{This}{this}\SetKwData{Up}{up}
\SetKwFunction{Union}{Union}\SetKwFunction{FindCompress}{FindCompress}
\SetKwInOut{Input}{Input}\SetKwInOut{Output}{Output}
\BlankLine
\Indm
\Input{A webgraph $G$ with $n$ nodes}
\Output{An array $pr$ of size $n$}
\Indp
\BlankLine
\emph{// Initialization}\;
\For{$i \leftarrow 1$ \KwTo $n$}{
$pr'\left[i\right]\leftarrow\infty$\;
$pr\left[i\right]\leftarrow\frac{1}{n}$\;
}
\emph{// Iterations}\;
\While{$|pr-pr'|\geq\epsilon$}{
$pr' \leftarrow pr$\;
\For{$i \leftarrow 1$ \KwTo $n$}{
$pr\left[i\right]\leftarrow\frac{1-d}{N}+d\sum_{j\in M\left(i\right)}\frac{pr'[j]}{L\left(j\right)}$\;
}
\emph{// Renormalization}\;
$s \leftarrow 0$\;
\For{$i\leftarrow 1$ \KwTo $n$}{
$s \leftarrow s + pr\left[i\right]$\;
}
\For{$i\leftarrow 1$ \KwTo $n$}{
$pr\left[i\right] \leftarrow pr\left[i\right] + \frac{1-s}{n}$\;
}
}
\BlankLine
\caption{Power method}\label{algo_power_method}
\end{algorithm}
\DecMargin{1em}

\section{Results}

Bla, bla, bla...


\section{Problems and solutions}

Bla, bla, bla...


\section{Non realised elements}

Bla, bla, bla...


\section{Conclusion}

Bla, bla, bla...


\end{document}
